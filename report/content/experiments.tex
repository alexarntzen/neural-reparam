
\section{Experiments}
\FloatBarrier
When dealing with neural networks there are a wide range of network structures and optimization algorithms. An in depth search for the best parameters will not be the focus of this chapter. The main focus of this chapter will be to investigate when reparametrization by neural networks is possible.

We also investigate whether the deviation from the optimal solution is optimization error or approximation error.  Thus by the results in chapter [0] we investigate how increasing the network size impacts the resulting error. This approach will still be somewhat ambiguous; since increasing the networks size simultaneously increases the dimension of the optimization problem. Also with increasing layers we have the vanishing gradient problem.[0]

\subsection{Curves from the same shape}
\FloatBarrier
\begin{figure}[t]\label{fig:curve_1}
    \begin{subfigure}[b]{0.5\textwidth}\label{fig:curve_1_c_1}
        \centering
        \includegraphics[width=\linewidth]{figures/curve_1/curve_c_1.pdf}
        \caption{\(c_1\)}
    \end{subfigure}
    \begin{subfigure}[b]{0.5\textwidth}\label{fig:curve_1_c_2}
        \centering
        \includegraphics[width=\linewidth]{figures/curve_1/curve_c_2.pdf}
        \caption{\(c_2\)}
    \end{subfigure}
    \begin{subfigure}[t]{0.5\textwidth}\label{fig:curve_1_q}
        \centering
        \includegraphics[width=\linewidth]{figures/curve_1/curve_q.pdf}
        \caption{\(q\)}
    \end{subfigure}
    \begin{subfigure}[t]{0.5\textwidth}\label{fig:curve_r}
        \centering
        \includegraphics[width=\linewidth]{figures/curve_1/curve_r.pdf}
        \caption{\(r\)}
    \end{subfigure}
    \caption{The trajectory of \(c_1, c_2 \in \text{Imm}(I, \R^2)\), \(q = Q(c_1)\) and \(r = Q(c_2)\).}
\end{figure}

\begin{figure}[t]\label{fig:curve_1_example}
    \begin{subfigure}[t]{0.5\textwidth}\label{fig:curve_1_solution}
        \centering
        \includegraphics[width=\linewidth]{figures/curve_1/curve_1_exp_4/curve_1_exp_4_5_0.pdf}
        \caption{The approximate optimal reparametrization and the analytical solution.}
    \end{subfigure}
    \begin{subfigure}[t]{0.5\textwidth}\label{fig:curve_1_history}
        \centering
        \includegraphics[width=\linewidth]{figures/curve_1/curve_1_exp_4/history_curve_1_exp_4_5.pdf}
        \caption{The cost function \(L(\theta)\) with each iteration.}
    \end{subfigure}
    \caption{The approximate optimal reparametrization solution to test problem (1) with the corresponding loss history.}
\end{figure}

\begin{figure}\label{fig:curve_1_parmas_eks}
    \begin{subfigure}[t]{0.5\textwidth}
        \centering
        \includegraphics[width=\linewidth]{figures/curve_1/curve_1_exp_4/neurons_error.pdf}
        \caption{The final cost \(E\) with the number of neurons in each hidden layer.}
        \label{fig:curve_1_neuron_error}
    \end{subfigure}
    \begin{subfigure}[t]{0.5\textwidth}
        \centering
        \includegraphics[width=\linewidth]{figures/curve_1/curve_1_exp_4/layer_error.pdf}
        \caption{Final cost \(E\) with the number of layers.}
        \label{fig:curve_1_layer_error}
    \end{subfigure}
    \caption{Result of ensemble training with different number of neurons and hidden layers. In Figure \ref{fig:curve_1_neuron_error} the number of layers was fixed at 2. In Figure \ref{fig:curve_1_layer_error} the number of neurons is fixed at 8 per hidden layer. The error bars denote a 80\% confidence interval found by bootstrapping.}
\end{figure}



\begin{tabular}{lrl}
    \toprule
    \(\bar{E} \) & Degrees of freedom & model         \\
    \midrule
    0.013143     & 34                 & ResNET        \\
    0.002253     & 9391               & ResNET        \\
    0.004567     & 30                 & Palais Layers \\
    0.000002     & 10000              & Palais Layers \\
    \bottomrule
\end{tabular}


Better structures and optimization algoritmhs for deeper networks.

\FloatBarrier
\subsection{Curves from different shapes}
\begin{figure}[t]\label{fig:curve_2}
    \begin{subfigure}[b]{0.5\textwidth}\label{fig:curve_2_c_1}
        \centering
        \includegraphics[width=\linewidth]{figures/curve_2/curve_c_1.pdf}
        \caption{\(c_1\)}
    \end{subfigure}
    \begin{subfigure}[b]{0.5\textwidth}\label{fig:curve_2_c_2}
        \centering
        \includegraphics[width=\linewidth]{figures/curve_2/curve_c_2.pdf}
        \caption{\(c_2\)}
    \end{subfigure}
    \begin{subfigure}[t]{0.5\textwidth}\label{fig:curve_2_q}
        \centering
        \includegraphics[width=\linewidth]{figures/curve_2/curve_q.pdf}
        \caption{\(q\)}
    \end{subfigure}
    \begin{subfigure}[t]{0.5\textwidth}\label{fig:curve_2_r}
        \centering
        \includegraphics[width=\linewidth]{figures/curve_2/curve_r.pdf}
        \caption{\(r\)}
    \end{subfigure}
    \caption{The trajectory of \(c_1, c_2 \in \text{Imm}(I, \R^2)\), \(q = Q(c_1)\) and \(r = Q(c_2)\).}
\end{figure}

\begin{figure}[t]\label{fig:curve_2_example}
    \begin{subfigure}[t]{0.5\textwidth}\label{fig:curve_2_solution}
        \centering
        \includegraphics[width=\linewidth]{figures/curve_2/curve_2_exp_2/curve_2_exp_2_1_0.pdf}
        \caption{The approximate optimal reparametrization and the analytical solution.}
    \end{subfigure}
    \begin{subfigure}[t]{0.5\textwidth}\label{fig:curve_2_history}
        \centering
        \includegraphics[width=\linewidth]{figures/curve_2/curve_2_exp_2/history_curve_2_exp_2_0.pdf}
        \caption{The cost function \(L(\theta)\) with each iteration.}
    \end{subfigure}
    \caption{The approximate solution to test problem (2) found by a neural network and the corresponding loss history.}
\end{figure}

\begin{figure}[t]\label{fig:curve_2_parmas_eks}
    \begin{subfigure}[t]{0.5\textwidth}
        \centering
        \includegraphics[width=\linewidth]{figures/curve_2/curve_2_exp_2/neurons_error.pdf}
        \caption{The final cost \(E\) with the number of neurons in each hidden layer.}
        \label{fig:curve_2_neuron_error}
    \end{subfigure}
    \begin{subfigure}[t]{0.5\textwidth}
        \centering
        \includegraphics[width=\linewidth]{figures/curve_2/curve_2_exp_2/layer_error.pdf}
        \caption{Final cost \(E\) with the number of layers.}
        \label{fig:curve_2_layer_error}
    \end{subfigure}
    \caption{Result of ensemble training for test problem (2) with different number of neurons and hidden layers. In Figure \ref{fig:curve_2_neuron_error} the number of layers was fixed at 2. In Figure \ref{fig:curve_2_layer_error} the number of neurons is fixed at 8 per hidden layer. The error bars denote a 80\% confidence interval found by bootstrapping.}
\end{figure}


Better structures and optimization algoritmhs for deeper networks.


\FloatBarrier
\subsection{Piecewise linear and piecewise constant}

\begin{figure}[t]\label{fig:curve_1_pc}
    \begin{subfigure}[t]{0.5\textwidth}\label{fig:curve_1_pc_q}
        \centering
        \includegraphics[width=\linewidth]{figures/curve_1/curve_q_pc.pdf}
        \caption{\(\bar q\)}
    \end{subfigure}
    \begin{subfigure}[t]{0.5\textwidth}\label{fig:curve_1_pc_r}
        \centering
        \includegraphics[width=\linewidth]{figures/curve_1/curve_r_pc.pdf}
        \caption{\(\bar r\)}
    \end{subfigure}
    \caption{The trajectory of \(\bar q\) and \(\bar r\) being piecewise constant interpolations of $q$ and $r$ from test problem (1).}
\end{figure}

\begin{figure}[t]\label{fig:curve_1_pc_pl_example}
    \begin{subfigure}[t]{0.5\textwidth}
        \centering
        \includegraphics[width=\linewidth]{figures/curve_1_pc/curve_pc_1/curve_pc_1_0_0.pdf}
        \caption{The approximate optimal reparametrization and the analytical solution.}
        \label{fig:curve_1_pc_solution}
    \end{subfigure}
    \begin{subfigure}[t]{0.5\textwidth}
        \centering
        \includegraphics[width=\linewidth]{figures/curve_1_pc/curve_pc_1/history_curve_pc_1_0.pdf}
        \caption{The cost function \(L(\theta)\) with each iteration.}
        \label{fig:curve_1_pc_history}
    \end{subfigure}
    \begin{subfigure}[t]{0.5\textwidth}
        \centering
        \includegraphics[width=\linewidth]{figures/curve_1_pl/curve_1_pl_exp_1/curve_1_pl_exp_1_1_0.pdf}
        \caption{The approximate optimal reparametrization and the analytical solution.}
        \label{fig:curve_1_pl_solution}
    \end{subfigure}
    \begin{subfigure}[t]{0.5\textwidth}
        \centering
        \includegraphics[width=\linewidth]{figures/curve_1_pl/curve_1_pl_exp_1/history_curve_1_pl_exp_1_1.pdf}
        \caption{The cost function \(L(\theta)\) with each iteration.}
        \label{fig:curve_1_pl_history}
    \end{subfigure}
    \caption{The approximate solutions to the piecewise constant (\ref{fig:curve_1_pc_solution}) and piecewise linear (\ref{fig:curve_1_pl_solution}) version of test problem (1). The approximate solution is compared to the analytical solution of test problem (1) and the corresponding training history is shown in the figure to the right.}
\end{figure}

\begin{figure}[t]\label{fig:curve_1_pl_eks}
    \begin{subfigure}[t]{0.5\textwidth}
        \centering
        \includegraphics[width=\linewidth]{figures/curve_1_pl/curve_1_pl_exp_1/neurons_error.pdf}
        \caption{The final cost \(E\) with the number of neurons in each hidden layer.}
        \label{fig:curve_1_pl_neuron_error}
    \end{subfigure}
    \begin{subfigure}[t]{0.5\textwidth}
        \centering
        \includegraphics[width=\linewidth]{figures/curve_1_pl/curve_1_pl_exp_1/layer_error.pdf}
        \caption{Final cost \(E\) with the number of layers.}
        \label{fig:curve_1_pl_layer_error}
    \end{subfigure}
    \caption{Result of ensemble training for the piecewise linear version of test problem (1) with different number of neurons and hidden layers. In Figure \ref{fig:curve_2_neuron_error} the number of layers was fixed at 2. In Figure \ref{fig:curve_2_layer_error} the number of neurons is fixed at 8 per hidden layer. The error bars denote a 80\% confidence interval found by bootstrapping.}
\end{figure}

Training of
Error convergence example 

\FloatBarrier
\subsection{Interpolated motion capture data}


\begin{figure}[t]\label{fig:curve_1_so3_example}
    \begin{subfigure}[t]{0.5\textwidth}
        \centering
        \includegraphics[width=\linewidth]{figures/curve_so3/pc_eks_2/plot_1_0.pdf}
        \caption{Solutions found by neural network training and dynamic programming.}
        \label{fig:curve_so3_pc_solution}
    \end{subfigure}a
    \begin{subfigure}[t]{0.5\textwidth}
        \centering
        \includegraphics[width=\linewidth]{figures/curve_so3/pc_eks_2/history_plot_1.pdf}
        \caption{The cost function \(L(\theta)\) with each iteration.}
        \label{fig:curve_so3_pc_history}
    \end{subfigure}
    \begin{subfigure}[t]{0.5\textwidth}
        \centering
        \includegraphics[width=\linewidth]{figures/curve_so3/pl_eks_6/plot_288_0.pdf}
        \caption{Solutions found by neural network training and dynamic programming.}
        \label{fig:curve_so3_pl_solution}
    \end{subfigure}
    \begin{subfigure}[t]{0.5\textwidth}
        \centering
        \includegraphics[width=\linewidth]{figures/curve_so3/pl_eks_6/history_plot_288.pdf}
        \caption{The cost function \(L(\theta)\) with each iteration.}
        \label{fig:curve_so3_pl_history}
    \end{subfigure}
    \caption{The approximate optimal reparametrizations of two curves representing motion capture data. Figure \ref{fig:curve_so3_pc_solution} shows the solution for the piecewise constant interpolation of the data, and Figure \ref{fig:curve_so3_pl_solution} shows the solution for the linearly interpolated curve. The approximate solutions are compared to the solution found by the dynamic programming algorithm and the corresponding training history is shown in the figure to the right.}
\end{figure}

\begin{figure}[t]\label{fig:curve_so3_pl_eks}
    \begin{subfigure}[t]{0.5\textwidth}
        \centering
        \includegraphics[width=\linewidth]{figures/curve_so3/pl_eks_6/neurons_error.pdf}
        \caption{The final cost \(E\) with the number of neurons in each hidden layer.}
        \label{fig:curve_so3_pl_neuron_error}
    \end{subfigure}
    \begin{subfigure}[t]{0.5\textwidth}
        \centering
        \includegraphics[width=\linewidth]{figures/curve_so3/pl_eks_6/layer_error.pdf}
        \caption{Final cost \(E\) with the number of layers.}
        \label{fig:curve_so3_pl_layer_error}
    \end{subfigure}
    \caption{Result of ensemble training for the piecewise linear version of the problem with curves generated by motion capture data with different number of neurons and hidden layers. In Figure \ref{fig:curve_2_neuron_error} the number of layers was fixed at 2. In Figure \ref{fig:curve_2_layer_error} the number of neurons is fixed at 8 per hidden layer. The error bars denote a 80\% confidence interval found by bootstrapping.}
\end{figure}


