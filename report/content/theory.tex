
\section{Theory}
\subsection{Riemannian geometry}

\subsection{Lie groups}
this subsection some basic Lie group properties are introduced. Based on \cite{celledoni2016}. Basic knowledge of smooth manifolds is assumed.
\begin{definition}[Lie group]
  A Lie group  \(G\) is a smooth manifold that is also a group, such that multiplication
  \begin{equation}
    \mu : G \times G \rightarrow G
  \end{equation}
  and inversion
  \begin{equation}
    i : G  \rightarrow G
  \end{equation}
  are smooth.
\end{definition}
Since both  multiplication and inversion are smooth we define multiplication by  \(g \in G\) as diffeomorphism.
\begin{definition}
  Let  \(G\) be a Lie group and  \(g \in G\). Then the \textbf{right translation} by  \(g\) is defined as
  \begin{equation}
    R_g(h):= hg \quad  \forall \ h  \in G.
  \end{equation}
\end{definition}
\begin{remark}
  in this text we use the \textbf{right} translation, but we could similarly define the left translation.
\end{remark}
Once translation is defined, the following property of vector fields follows.
\begin{definition}
  A vector field  \(X\) on a Lie group  \(G\) is \textbf{right invariant} if it is invariant under the pushforward of all right translation. That is is
  \begin{equation}
    (r_g)_*X = X  \quad \forall g \in G
  \end{equation}
\end{definition}
An invariant vector field  \(X\) on a Lie group  \(G\) is thus uniquely defined by the vector field's value at the identity element  \(e\). Similarly, for all vectors  \(\xi \in T_eG\) we can define an invariant vector field   \(X^{\xi}\) on  \(G\).
\begin{definition}[Lie algebra]
  Let  \(e\) be the identity element of the Lie group  \(G\). Then the \textbf{Lie algebra}  \(\mathfrak{g}\) is the vector space  \(T_{e}G\) together with a lie bracket as bilinear product.
\end{definition}
\subsection{The exponential map}

\begin{definition}[Exponential map]
  Let  \(\mathfrak{g}\) be the Lie algebra of the Lie group  \(G\) with identity  \(e\). For all  \(\xi \in \mathfrak{g}\) let  \(\gamma^{\xi}\) be the integral curve of the right invariant vector field  \(X^{\xi}\) where  \(X^\xi (e)=\xi\) and  \(\gamma^\xi(0) = e\). The \textbf{exponential map} is the map  \(\exp : \mathfrak{g} \rightarrow G\) such that  \(\exp(\xi) = \gamma^\xi(1)\).
\end{definition}
% TODO: make an argumet for why it is well defined
\begin{proposition}
  Let  \(X\) be a right invariant vector field on the Lie group  \(G\) with identity  \(e\). Then the flow of  \(X\),  \(\phi_t\) is given by  \(\phi_t(g) = \exp(tX(e))g\) for all  \(g\) in  \(G\).
\end{proposition}
\begin{proof}
  Denote  \(\gamma^{\xi}\) as the integral curve of an  invariant vector field  \(X^{\xi}\)such that  \(\dot{\gamma}^{\xi}(0) = \xi \) and  \(\gamma^\xi(0)=e\). Then  \(\gamma^{t\xi}(0) =  \gamma^{\xi}(0)=e\) and
  \begin{equation}
    \frac{d}{ds}\gamma^{\xi}(ts) = t\dot\gamma^{\xi}(st)=tX^\xi(\gamma^{\xi}(ts)) = t D R_{\gamma^\xi(ts)} \vert_{e} \xi = X^{t\xi}(\gamma^{\xi}(ts)).
  \end{equation}
  So  \(\gamma^\xi(ts)\) is the integral curve of  \(X^{t\xi}\) and  \(\gamma^\xi(ts)=\gamma^{t\xi}(s)\). Therefore,   \(\exp(t\xi) = \gamma^{\xi}(t)\).

  Now, we can show that  \(\exp(tX(e))g\) is the integral curve of  \(X^{\xi}\) starting at  \(g\). Firstly,  \(\exp(0)g = \gamma^{t\xi}(0)g=g\). Then, computing the derivative we have
  \begin{equation}
    \frac{d}{dt} \exp(t\xi)g = D R_g \vert_{\gamma^\xi(t)} X^\xi(\gamma^\xi(t)) = X^\xi(\exp(t\xi)g).
  \end{equation}
  By uniqueness of integral curves, the result follows.
\end{proof}
% TODO: show that the flow is defined for all times t 
\begin{remark}
  If  \(G\) is a subgroup of  \(GL(n)\) then it can be shown that the  \(\exp(A)\) is the matrix exponential  \(e^{A}\).
\end{remark}


The exponential map also has a unique inverse in a neigboorhodd.
\begin{definition}[Logatimic derivative]
  \(\delta^r(\gamma)\)
\end{definition}
\begin{proposition}
  The right logarithmic derivative is the inverse of the exponential map?
\end{proposition}
local isometry
isometry



