\subsection{The square root velocity transform}
A method for imposing a metric on \(\mathcal{P}\), is to transform the curves with a diffeomorphism to another Riemannian manifold, then computing the corresponding pullback metric on \(\mathcal{P}\). A popular such transformation, introduced in \cite{srivastava2011_srvt}, is the \emph{square root velocity transform} (SRVT), which is given by
\begin{equation}\label{eq:SRVT}
  R :\mathcal{P} \rightarrow C^{\infty}(I, R^d \setminus \{0\}), \quad c \mapsto \frac{c'}{\sqrt{\vert c' \vert}}.
\end{equation}

The SRVT has several properties which is not proved here; for a thorough introduction see \cite{bruveris1016_srvtexample,bauer2014_rprop}. Firstly, the SRVT is invariant under translation and therefore not injective on \(\mathcal{P}\). It will however, be a diffeomorphism between the space of curves with zero starting position \(\mathcal{P}_* : = \{c \in \mathcal{P}: c(0) = 0\} \) and \(C^{\infty}(I, \R^d \setminus \{0\})\). Between these spaces, the SRVT also has an analytical inverse on the form
\begin{equation*}
  R^{-1}(q)(t) = \int_0 ^t q \vert q\vert \,d\tau.
\end{equation*}
Furthermore, imposing the \(L^2\) inner product on\((C^{\infty}(I, R^d), \langle \cdot , \cdot \rangle_{L^2} )\) gives an inner product space. Thus, the SRVT imposes a pullback distance metric and a Riemannian metric on \(\mathcal{P}_*\). The pullback distance metric for the two curves \(c_1, c_2 \in \mathcal{P}_*\) will be given by
\begin{equation}\label{eq:sob_pullback_dist}
  d_{\mathcal{P}_*}(c_1, c_2) = \norm{R(c_1) - R(c_2)}_{L^2}.
\end{equation}
Similarly, the pullback Riemannian metric on \(\mathcal{P}_*\) will be a \emph{first order Sobolev metric}, defined by the arch length integral
\begin{equation*}
  G_c(h,k) = \int_I \langle D_s h^\perp ,D_s^\perp k \rangle+\frac{1}{4}\langle D_s h,v\rangle \langle D_s k,v\rangle \,ds,
\end{equation*}
where \(ds = \vert c' \vert\,dt\), \(v = D_s c = \frac{c'}{\vert c'\vert}\) is the curve of unit length tangents of \(c\), and \(D_s h^\perp = D_s h  - \langle D_s h,v\rangle v\) is the projection of \(D_s h\) onto the space of curves orthogonal to \(v\). \todo{remark on properties of this metric}

The space \(C^{\infty}(I, \R^d \setminus \{0\}) \) is flat, but not convex. Therefore, when the SRVT form of two curves \(R(c_1), R(c_2)\) belong to the same convex subset of \(C^{\infty}(I, \R^d \setminus \{0\})\). Then the  geodesics from \(c_1\) to \(c_2\) is given by
\begin{equation*}
  \gamma(t) = R^{-1}[R(c_2)t +  R(c_1)(1-t)],
\end{equation*}
and the length of \(\gamma\) will correspond to the distance given by \(d_{\mathcal{P}_*}\). For transformed curves not connected by a straight line there are two cases. Either the geodesic of the metric \(G\) will not exist, or the geodesic distance induced by \(G\) will not correspond to the  distance \(d_{\mathcal{P}_*}\). When \(d \geq 3\),  we can make lines in \(C^{\infty}(I, \R^d )\) into paths in \(C^{\infty}(I, \R^d \setminus \{0\}\) by arbitrary small perturbations. Thus, the distance given by \(d_{\mathcal{P}_*}\) will be the induced geodesic distance of \(G\). As a remedy for \(d \leq 2 \), it is possible to consider geodesic completions of \(\mathcal{P}_* \) as discussed in \cite{bruveris1016_srvtexample} and more generally in \cite{bruveris2014_geocomp}.


To extend the Riemannian geometry of \(\mathcal{P}\) to the shape space, we note that the SRVT has the equivariance property:
\begin{equation*}
  R(c \circ \varphi) = \sqrt{\varphi'}R(c) \circ \varphi \quad \forall \ \varphi \in \text{Diff}^+, c \in \mathcal{P}.
\end{equation*}
Therefore, we deduce by integration or \cite[Theomrem 3.1]{bauer2014_rprop} that the metrics \(G\) and \(d_{\mathcal{P}_*}\) are reparametrization invariant. The reparametrization invariant distance can then be extended to \(\mathcal{S}_* : = \{[c] \in \mathcal{S}: c(0)=0\}\) by \eqref{eq:distance_inf}. Consequently, finding distances and geodesics between two shapes \([c_1], [c_2]\in \mathcal{S}_*\) has been reduced to finding a reparametrization \(\varphi \in \DiffI\) that minimizes
\begin{equation} \label{eq:srvt_reparam}
  \norm{R(c_1) - \sqrt{\varphi'} R(c_2)\circ \varphi}_{L^2}.
\end{equation}

The optimal solution to the optimization problem above will not necessarily exist in \(\DiffI\). Examples of curves that lead to optimal reparametrization outside \(\DiffI\) can be found in \cite[p.11]{bauer2015why} and \cite[Section 3.2]{woien2019}. One problem is that optimal solutions might have vanishing derivatives. Thus, the optimal solution is not guaranteed to be a diffeomorphism. Still, there are metrics on \(\mathcal{P}\) for which the optimal diffeomorphism is guaranteed as shown in \cite{bauer2014overview}. \citeauthor{bruveris1016_srvtexample} \cite{bruveris1016_srvtexample} instead expands the search space to
\begin{equation*}
  \overline \Gamma = \{\gamma : I \rightarrow I : \gamma \text{\ abs. \ cont. }, \gamma(0) = 0, \gamma(1) = 1, \gamma' \geq 0 \ \text{a.e.}\}.
\end{equation*}
The existence problem is then remedied by showing that for curves \(c_1, c_2 \in C^2(I,\R^d)\) there exists optimal reparametrizations \(\varphi_1^*, \varphi_2^* \in \overline{ \Gamma}\) that minimizes
\begin{equation*}
  d_{\mathcal{P}_*}(c_1 \circ \varphi_1, c_2 \circ \varphi2).
\end{equation*}

\subsection{Shape analysis on Lie groups}\label{subsec:shape-lie}
In \cite{celledoni2016} an extension of the SRVT to Lie groups, transforming to Lie algebra.
\todo{definitions and properties, diffeomorphism, pullback}
\todo{motion capture data, interpolation, SRV form}

The SRV form of the motion capture data will then be the piecewise constant curve such that for \(t \in [t_i, t_{i+}\)
\begin{equation*}
  q(t) = \frac{\eta_i}{\sqrt{\norm{\eta_i}}},
\end{equation*}
where
\begin{equation*}
  \eta_i = \frac{\log(c_{i+1}c_{i}^T)}{t_{i+1} -  t_{i}}.
\end{equation*}