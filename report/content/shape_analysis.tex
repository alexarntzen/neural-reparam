\section{Shape analysis}
\subsection{Definitions and motivation}
To start the problem of analyzing curves we fist introduce the space of curves considered in this project.  To that end, we will follow the approach given by \citeauthor{bauer2015why} in \cite{bauer2015why}. The curves we will consider will be the space of all immersions from the interval  \(I = [0, 1]\) to  \(\R^d\)
\begin{equation}
  \mathcal{P}:= \text{Imm}(I, \R^d) = \{c \in C^{\infty}(I, \R^d) :  c'(t) \neq  0 \ \forall \ t \in I  \}.
\end{equation}
This space \(\mathcal{P}\) is generally called the \emph{pre-shape space}. Although we will restrict ourselves to the study of open curves, analysis of closed curves \(\text{Imm}(S^1, \R^d)\) and surfaces \(\text{Imm}(I \times I, \R^d)\) has also been investigated with a similar framework \cite{bauer2014overview}.

Now we would like to study shapes independent of parametrization. More precisely, we consider the space of all reparametrizations \(\varphi\) as the group of orientation preserving diffeomorphisms of \(I\)
\begin{equation}
  \text{Diff}^+(I) = \{\varphi \in C^{\infty}(I,I): \varphi(0) = 0, \varphi(1) = 1, \varphi'(t) > 0 \ \forall \ t \in I \}.
\end{equation}
Furthermore, for a reparametrization  \(\varphi \) and a curve  \(c\) a group action can be defined by  \(c \circ \varphi\). Thus we can define the space of parameterized curves \(\mathcal{S}\) as the quotient space under this group action,
\begin{equation}
  \mathcal{S} := \mathcal{P} \ / \ {\text{Diff}^+(I)}.
\end{equation}
Notably, space  \(\mathcal{S}\) is generally called the \emph{shape space} and contains with elements called \emph{shapes}. Furthermore, the space \(\mathcal{S}\) is itself an infinite-dimensional manifold, with the projection map being a submersion. 

\subsection{Distance on shape space}
To find a distance function \(d_{\mathcal{S}}\) on \(\mathcal{S}\) we start with distance on \(\mathcal{\mathcal{P}}\) that is \emph{reparametrization invariant}. That is, \(d_\mathcal{P} : \mathcal{P} \times P \rightarrow (0, \infty)\) has the property that for all  \(c_1, c_2 \in \mathcal{P}\)
\begin{equation}
  d_{\mathcal{P}}(c_1, c_2)=d_{\mathcal{P}}(c_1 \circ \varphi, c_2\circ \varphi) \quad \forall \varphi \in \text{Diff}^+.
\end{equation}

A reparametrization invariant distance function on \(\mathcal{P}\) will not necessarily be a distance function on \(S\) since  that distance between representatives of each shape has to be the same. Thus, a distance function on \(\mathcal{S}\) will we defined as. 
\begin{equation}
  d_\mathcal{S} ([c_1],[c_2]) := \inf_{\varphi \in \text{Diff}^+}{  d_{\mathcal{P}}(c_1,c_2 \circ \varphi)},
  \label{eq:distance_inf}
\end{equation} 
where  \([c_1]\) and  \([c_2]\) are the shapes with reparametrizations \(c_1\) and  \(c_2\). 

\subsection{Riemannian metrics on shape space}
As remarked in \cite{bauer2014overview}, the the shape space \(\mathcal{S}\) is inherently non-linear. Therefore we impose a \emph{reparametrization invariant Riemannian metric}  on \(\mathcal{P}\). That is, a Riemannian metric \(G\) such that for all \( c \in \mathcal{P}\) and \(h,k \in  T_c \mathcal{P} = C^\infty(I, \R^d) \)
\begin{equation}
  G_c(h,k) = G_{c \circ \varphi}(h\circ \varphi, k \circ \varphi) \quad \forall \varphi \in \text{Diff}^+.
\end{equation}
A reparametrization invariant Riemannian metric will then induce a reparametrization invariant  geodesic distance given by 
\begin{equation}
  d_\mathcal{P}(c_1, c_2) = \inf_{
    \substack{
      \gamma \in C^{\infty}([0,1], \mathcal{P}) \\ 
      \gamma(0) = c_1 \\
      \gamma(1) = c_2
    }
  } \int_0^1 \sqrt{G_{\gamma(t)}(\gamma'(t),\gamma'(t))} \, dt.
\end{equation}
Then, a distance function on \(\mathcal{S}\) can then be obtained by \eqref{eq:distance_inf}. The Riemannian metric on \(\mathcal{P}\) also defines a unique Riemannian metric on \(\mathcal{S}\) such that the projection map from \(\mathcal{P}\) to \(\mathcal{S}\) is a Riemannian submersion \cite[Section 6]{bauer2014_rprop}. 

An obvious metric on \(\mathcal{P}\) is the \(L^2\) metric  given by 
\begin{equation}
  G_c(h,k) = \int_I \langle h, k\rangle \vert c'\vert \,dt. 
\end{equation}
However, as shown in \cite{michor2003vanishingl2} and \cite{michor2004vanishing_generalized} the \(L^2\) metric induces a vanishing distance on shape space \(\mathcal{S}\). This means that for any \(c_1, c_2 \in \mathcal{P}\) we can construct curves \(\gamma \in C^{\infty}([0,1],\mathcal{P})\) starting at \(c_1\) ending at \(c_2 \circ \varphi\) of arbitrary short length. As a conclusion the \(L^2\) metric is useless for comparing shapes, and so we must find another metric. 