\section{Shape analysis}
\subsection{Definitions and motivation}
To start the problem of analyzing curves we fist introduce the space of curves considered in this project.  To that end, we will follow the approach given by \citeauthor{bauer2015why} in \cite{bauer2015why}. The curves we will consider will be the space of all immersions from the interval  \(I = [0, 1]\) to  \(\R^d\)
\begin{equation*}
  \mathcal{P}:= \text{Imm}(I, \R^d) = \{c \in C^{\infty}(I, \R^d) :  c'(t) \neq  0 \ \forall \ t \in I  \}.
\end{equation*}
This space \(\mathcal{P}\) is generally called the \emph{pre-shape space}. Although we will restrict ourselves to the study of open curves, analysis of closed curves \(\text{Imm}(S^1, \R^d)\) and surfaces \(\text{Imm}(I \times I, \R^d)\) has also been investigated with a similar framework \cite{bauer2014overview}.

We would like to study shapes independent of parametrization. More precisely, we consider the space of all reparametrizations \(\varphi\) as the group of orientation preserving diffeomorphisms of \(I\)
\begin{equation*}
  \text{Diff}^+(I) = \{\varphi \in C^{\infty}(I,I): \varphi(0) = 0, \varphi(1) = 1, \varphi'(t) > 0 \ \forall \ t \in I \}.
\end{equation*}
Furthermore, for a reparametrization  \(\varphi \) and a curve  \(c\), a group action can be defined by  \(c \circ \varphi\). Thus, we define the space of parameterized curves \(\mathcal{S}\) as the quotient space under this group action,
\begin{equation*}
  \mathcal{S} := \mathcal{P} \ / \ {\text{Diff}^+(I)}.
\end{equation*}
Notably, space  \(\mathcal{S}\) is generally called the \emph{shape space} and contains with elements called \emph{shapes}. Furthermore, the space \(\mathcal{S}\) is itself an infinite-dimensional manifold, with the projection map being a submersion.

\subsection{Distance on shape space}
To find a distance function \(d_{\mathcal{S}}\) on \(\mathcal{S}\) we start with a distance on \(\mathcal{\mathcal{P}}\) that is \emph{reparametrization invariant}. That is, \(d_\mathcal{P} : \mathcal{P} \times P \rightarrow (0, \infty)\) has the property that for all  \(c_1, c_2 \in \mathcal{P}\)
\begin{equation*}
  d_{\mathcal{P}}(c_1, c_2)=d_{\mathcal{P}}(c_1 \circ \varphi, c_2\circ \varphi) \quad \forall \varphi \in \text{Diff}^+.
\end{equation*}

A reparametrization invariant distance function on \(\mathcal{P}\) will not necessarily be a distance function on \(\mathcal{S}\) since  that distance between representatives of each shape has to be the same. Thus, a distance function on \(\mathcal{S}\) will we defined as
\begin{equation*}
  d_\mathcal{S} ([c_1],[c_2]) := \inf_{\varphi \in \text{Diff}^+}{  d_{\mathcal{P}}(c_1,c_2 \circ \varphi)},
  \label{eq:distance_inf}
\end{equation*}
where  \([c_1]\) and  \([c_2]\) are the shapes with reparametrizations \(c_1\) and  \(c_2\).

\subsection{Riemannian metrics on shape space}
As remarked in \cite{bauer2014overview}, the the shape space \(\mathcal{S}\) is inherently non-linear. Therefore we impose a \emph{reparametrization invariant Riemannian metric}  on \(\mathcal{P}\). That is, a Riemannian metric \(G\) such that for all \( c \in \mathcal{P}\) and \(h,k \in  T_c \mathcal{P} = C^\infty(I, \R^d) \)
\begin{equation*}
  G_c(h,k) = G_{c \circ \varphi}(h\circ \varphi, k \circ \varphi) \quad \forall \varphi \in \text{Diff}^+.
\end{equation*}
A reparametrization invariant Riemannian metric will then induce a reparametrization invariant  geodesic distance given by
\begin{equation*}
  d_\mathcal{P}(c_1, c_2) = \inf_{
    \substack{
      \gamma \in C^{\infty}([0,1], \mathcal{P}) \\
      \gamma(0) = c_1 \\
      \gamma(1) = c_2
    }
  } \int_0^1 \sqrt{G_{\gamma(t)}(\gamma'(t),\gamma'(t))} \, dt.
\end{equation*}
Then, a distance function on \(\mathcal{S}\) can then be obtained by \eqref{eq:distance_inf}. The Riemannian metric on \(\mathcal{P}\) also defines a unique Riemannian metric on \(\mathcal{S}\) such that the projection map from \(\mathcal{P}\) to \(\mathcal{S}\) is a Riemannian submersion \cite[Section 6]{bauer2014_rprop}.

An obvious metric on \(\mathcal{P}\) is the \(L^2\) metric  given by
\begin{equation*}
  G_c(h,k) = \int_I \langle h, k\rangle \vert c'\vert \,dt.
\end{equation*}
However, as shown in \cite{michor2003vanishingl2} and \cite{michor2004vanishing_generalized} the \(L^2\) metric induces a vanishing distance on shape space \(\mathcal{S}\). This means that for any \(c_1, c_2 \in \mathcal{P}\) we can construct curves \(\gamma \in C^{\infty}([0,1],\mathcal{P})\) starting at \(c_1\) ending at \(c_2 \circ \varphi\) of arbitrary short length. As a conclusion the \(L^2\) metric is useless for comparing shapes, and so we must find another metric.

\subsection{The square root velocity transform}
A method for imposing a metric on \(\mathcal{P}\), is to transform the curves with a diffeomorphism to another Riemannian manifold, then computing the corresponding pullback metric on \(\mathcal{P}\). A popular such transformation, introduced in \cite{srivastava2011_srvt}, is the \emph{square root velocity transform} (SRVT), which is given by
\begin{equation}\label{eq:SRVT}
  R :\mathcal{P} \rightarrow C^{\infty}(I, R^d \setminus \{0\}), \quad c \mapsto \frac{c'}{\sqrt{\vert c' \vert}}.
\end{equation}

The SRVT has several properties which is not proved here; for a thorough introduction see \cite{bruveris1016_srvtexample,bauer2014_rprop}. Firstly, the SRVT is invariant under translation and therefore not injective on \(\mathcal{P}\). It will however, be a diffeomorphism between the space of curves with zero starting position \(\mathcal{P}_* : = \{c \in \mathcal{P}: c(0) = 0\} \) and \(C^{\infty}(I, \R^d \setminus \{0\})\). Between these spaces, the SRVT also has an analytical inverse on the form
\begin{equation*}
  R^{-1}(q)(t) = \int_0 ^t q \vert q\vert \,d\tau.
\end{equation*}
Furthermore, imposing the \(L^2\) inner product on\((C^{\infty}(I, R^d), \langle \cdot , \cdot \rangle_{L^2} )\) gives an inner product space. Thus, the SRVT imposes a pullback distance metric and a Riemannian metric on \(\mathcal{P}_*\). The pullback distance metric for the two curves \(c_1, c_2 \in \mathcal{P}_*\) will be given by
\begin{equation}\label{eq:sob_pullback_dist}
  d_{\mathcal{P}_*}(c_1, c_2) = \norm{R(c_1) - R(c_2)}_{L^2}.
\end{equation}
Similarly, the pullback Riemannian metric on \(\mathcal{P}_*\) will be a \emph{first order Sobolev metric}, defined by the arch length integral
\begin{equation*}
  G_c(h,k) = \int_I \langle D_s h^\perp ,D_s^\perp k \rangle+\frac{1}{4}\langle D_s h,v\rangle \langle D_s k,v\rangle \,ds,
\end{equation*}
where \(ds = \vert c' \vert\,dt\), \(v = D_s c = \frac{c'}{\vert c'\vert}\) is the curve of unit length tangents of \(c\), and \(D_s h^\perp = D_s h  - \langle D_s h,v\rangle v\) is the projection of \(D_s h\) onto the space of curves orthogonal to \(v\). \todo{remark on properties of this metric}

The space \(C^{\infty}(I, \R^d \setminus \{0\}) \) is flat, but not convex. Therefore, when the SRVT form of two curves \(R(c_1), R(c_2)\) belong to the same convex subset of \(C^{\infty}(I, \R^d \setminus \{0\})\). Then the  geodesics from \(c_1\) to \(c_2\) is given by
\begin{equation*}
  \gamma(t) = R^{-1}[R(c_2)t +  R(c_1)(1-t)],
\end{equation*}
and the length of \(\gamma\) will correspond to the distance given by \(d_{\mathcal{P}_*}\). For transformed curves not connected by a straight line there are two cases. Either the geodesic of the metric \(G\) will not exist, or the geodesic distance induced by \(G\) will not correspond to the  distance \(d_{\mathcal{P}_*}\). When \(d \geq 3\),  we can make lines in \(C^{\infty}(I, \R^d )\) into paths in \(C^{\infty}(I, \R^d \setminus \{0\}\) by arbitrary small perturbations. Thus, the distance given by \(d_{\mathcal{P}_*}\) will be the induced geodesic distance of \(G\). As a remedy for \(d \leq 2 \), it is possible to consider geodesic completions of \(\mathcal{P}_* \) as discussed in \cite{bruveris1016_srvtexample} and more generally in \cite{bruveris2014_geocomp}.


To extend the Riemannian geometry of \(\mathcal{P}\) to the shape space, we note that the SRVT has the equivariance property:
\begin{equation*}
  R(c \circ \varphi) = \sqrt{\varphi'}R(c) \circ \varphi \quad \forall \ \varphi \in \text{Diff}^+, c \in \mathcal{P}.
\end{equation*}
Therefore, we deduce by integration or \cite[Theomrem 3.1]{bauer2014_rprop} that the metrics \(G\) and \(d_{\mathcal{P}_*}\) are reparametrization invariant. The reparametrization invariant distance can then be extended to \(\mathcal{S}_* : = \{[c] \in \mathcal{S}: c(0)=0\}\) by \eqref{eq:distance_inf}. Consequently, finding distances and geodesics between two shapes \([c_1], [c_2]\in \mathcal{S}_*\) has been reduced to finding a reparametrization \(\varphi \in \DiffI\) that minimizes
\begin{equation} \label{eq:srvt_reparam}
  \norm{R(c_1) - \sqrt{\varphi'} R(c_2)\circ \varphi}_{L^2}.
\end{equation}

The optimal solution to the optimization problem above will not necessarily exist in \(\DiffI\). Examples of curves that lead to optimal reparametrization outside \(\DiffI\) can be found in \cite[p.11]{bauer2015why} and \cite[Section 3.2]{woien2019}. One problem is that optimal solutions might have vanishing derivatives. Thus, the optimal solution is not guaranteed to be a diffeomorphism. Still, there are metrics on \(\mathcal{P}\) for which the optimal diffeomorphism is guaranteed as shown in \cite{bauer2014overview}. \citeauthor{bruveris1016_srvtexample} \cite{bruveris1016_srvtexample} instead expands the search space to
\begin{equation*}
  \overline \Gamma = \{\gamma : I \rightarrow I : \gamma \text{\ abs. \ cont. }, \gamma(0) = 0, \gamma(1) = 1, \gamma' \geq 0 \ \text{a.e.}\}.
\end{equation*}
The existence problem is then remedied by showing that for curves \(c_1, c_2 \in C^2(I,\R^d)\) there exists optimal reparametrizations \(\varphi_1^*, \varphi_2^* \in \overline{ \Gamma}\) that minimizes
\begin{equation*}
  d_{\mathcal{P}_*}(c_1 \circ \varphi_1, c_2 \circ \varphi2).
\end{equation*}

\subsection{Shape analysis on Lie groups}\label{subsec:shape-lie}
In \cite{celledoni2016} an extension of the SRVT to Lie groups, transforming to Lie algebra.
\todo{definitions and properties, diffeomorphism, pullback}
\todo{motion capture data, interpolation, SRV form}

The SRV form of the motion capture data will then be the piecewise constant curve such that for \(t \in [t_i, t_{i+}\)
\begin{equation*}
  q(t) = \frac{\eta_i}{\sqrt{\norm{\eta_i}}},
\end{equation*}
where
\begin{equation*}
  \eta_i = \frac{\log(c_{i+1}c_{i}^T)}{t_{i+1} -  t_{i}}.
\end{equation*}