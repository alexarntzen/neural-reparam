\section{Shape analysis}\label{sec:shape-analysis}
\subsection{Definitions and motivation}
To start the problem of analyzing curves we fist introduce the space of curves considered in this project. To that end, we will follow the approach given by \citeauthor{bauer2015why} in \cite{bauer2015why}. The curves we will consider will be the space of all immersions from the interval  \(I = [0, 1]\) to  \(\R^d\)
\begin{equation*}
  \mathcal{P}:= \text{Imm}(I, \R^d) = \{c \in C^{\infty}(I, \R^d) :  c'(t) \neq  0 \ \forall \ t \in I  \}.
\end{equation*}
This space \(\mathcal{P}\) is generally called the \emph{pre-shape space}. Although we will restrict ourselves to the study of open curves, analysis of closed curves \(\text{Imm}(S^1, \R^d)\) and surfaces \(\text{Imm}(I \times I, \R^d)\) has also been investigated with a similar framework \cite{bauer2014overview}.

We would like to study shapes, i.e. curves independent on their parametrization. More precisely, we consider the space of all reparametrizations \(\varphi\) as the group of orientation preserving diffeomorphisms of \(I\)
\begin{equation*}
  \text{Diff}^+(I) = \{\varphi \in C^{\infty}(I,I): \varphi(0) = 0, \varphi(1) = 1, \varphi'(t) > 0 \ \forall \ t \in I \}.
\end{equation*}
The group action 
\begin{equation}
  \psi : \DiffI \times \mathcal{P} \rightarrow \mathcal{P},\quad \psi(\phi, c) \mapsto c \circ \phi 
\end{equation}
can then be used to construct the \emph{shape space} \(\mathcal{S}\) as the quotient space under this group action,
\begin{equation*}
  \mathcal{S} := \mathcal{P} \ / \ {\text{Diff}^+(I)}.
\end{equation*}
The space \(\mathcal{S}\) is also an infinite-dimensional manifold with elements called \emph{shapes}.
% being a submersion. \todo{fix this}

\subsection{Distance on shape space}
To find a distance function \(d_{\mathcal{S}}\) on \(\mathcal{S}\) we start with a distance on \(\mathcal{\mathcal{P}}\) that is \emph{reparametrization invariant}. That is, \(d_\mathcal{P} : \mathcal{P} \times P \rightarrow (0, \infty)\) has the property that for all  \(c_1, c_2 \in \mathcal{P}\)
\begin{equation*}
  d_{\mathcal{P}}(c_1, c_2)=d_{\mathcal{P}}(c_1 \circ \varphi, c_2\circ \varphi) \quad \forall \varphi \in \text{Diff}^+.
\end{equation*}

A reparametrization invariant distance function on \(\mathcal{P}\) is not enough to define a distance function on \(\mathcal{S}\) directly. Instead, a distance function on \(\mathcal{S}\) is given by 
\begin{equation}
  d_\mathcal{S} ([c_1],[c_2]) := \inf_{\varphi \in \text{Diff}^+}{  d_{\mathcal{P}}(c_1,c_2 \circ \varphi)},
  \label{eq:distance_inf}
\end{equation}
where  \([c_1]\) and  \([c_2]\) are the shapes with representatives \(c_1\) and  \(c_2\). Furthermore, this distance function on \(\mathcal{S}\) is well defined by \cite{celledoni2016}[Lemma 3.4].


\subsection{Riemannian metrics on shape space}
As remarked in \cite{bauer2014overview}, the the shape space \(\mathcal{S}\) is inherently non-linear. Therefore we impose a Riemannian metric  on \(\mathcal{P}\). Furthermore, we require that the Riemannian metric is reparametrization invariant. That is, a Riemannian metric \(G\) such that for all \( c \in \mathcal{P}\) and \(h,k \in  T_c \mathcal{P} = C^\infty(I, \R^d) \)
\begin{equation*}
  G_c(h,k) = G_{c \circ \varphi}(h\circ \varphi, k \circ \varphi) \quad \forall \varphi \in \text{Diff}^+.
\end{equation*}
It can then be shown that a reparametrization invariant Riemannian metric will  induce a reparametrization invariant distance on \(\mathcal{P}\) given by
\begin{equation*}
  d_\mathcal{P}(c_1, c_2) = \inf_{
    \substack{
      \gamma \in C^{\infty}([0,1], \mathcal{P}) \\
      \gamma(0) = c_1 \\
      \gamma(1) = c_2
    }
  } \int_0^1 \sqrt{G_{\gamma(t)}(\gamma'(t),\gamma'(t))} \, dt.
\end{equation*}
Therefore, a well defined distance on \(\mathcal{S}\) can be obtained by \eqref{eq:distance_inf}. 
%The Riemannian metric on \(\mathcal{P}\) also defines a unique Riemannian metric on \(\mathcal{S}\) such that the projection map from \(\mathcal{P}\) to \(\mathcal{S}\) is a Riemannian submersion \cite[Section 6]{bauer2014_rprop}.\todo{fix this}

An obvious metric on \(\mathcal{P}\) is the \(L^2\) metric  given by
\begin{equation*}
  G_c(h,k) = \int_I \langle h, k\rangle \vert c'\vert \,dt.
\end{equation*}
However, as shown in \cite{michor2003vanishingl2} and \cite{michor2004vanishing_generalized} the \(L^2\) metric induces a vanishing distance on shape space \(\mathcal{S}\). This means that for any \(c_1, c_2 \in \mathcal{P}\) we can construct curves \(\gamma \in C^{\infty}([0,1],\mathcal{P})\) starting at \(c_1\) ending at \(c_2 \circ \varphi\) of arbitrary short length. As a conclusion the \(L^2\) metric is useless for comparing shapes, and so we are bound to fined another metric.

\subsection{The square root velocity transform}
We now restrict our study to curves starting at zero define the space \(\mathcal{P}_* : = \{c \in \mathcal{P}: c(0) = 0\}\). A method for imposing a metric on \(\mathcal{P}_*\), is to transform the curves with a diffeomorphism to another Riemannian manifold, then computing the corresponding pullback metric on \(\mathcal{P}\). A popular such transformation, introduced in \cite{srivastava2011_srvt}, is the \emph{square root velocity transform} (SRVT), which is given by
\begin{equation}\label{eq:SRVT}
  R :\mathcal{P} \rightarrow C^{\infty}(I, R^d \setminus \{0\}), \quad c \mapsto \frac{c'}{\sqrt{\vert c' \vert}}.
\end{equation}

The SRVT has several properties which are not proved here; for a thorough introduction see \cite{bruveris1016_srvtexample,bauer2014_rprop}. Firstly, the SRVT is invariant under translation and therefore not injective on \(\mathcal{P}\). It will however, be a diffeomorphism between \(\mathcal{P}\) nd \(C^{\infty}(I, \R^d \setminus \{0\})\)\cite{bruveris1016_srvtexample}[Theorem 1]. Between these spaces, the SRVT also has an analytical inverse on the form
\begin{equation*}
  R^{-1}(q)(t) = \int_0 ^t q \vert q\vert \,d\tau.
\end{equation*}
Furthermore, imposing the \(L^2\) inner product on\((C^{\infty}(I, R^d), \langle \cdot , \cdot \rangle_{L^2} )\) gives an inner product space. Thus, the SRVT imposes a pullback distance metric  
\begin{equation*}
  d_{\mathcal{P}_*}(c_1, c_2) = \norm{R(c_1) - R(c_2)}_{L^2}, 
\end{equation*}
and a pullback Riemannian metric, 
\begin{equation*}
  G_c(h,k) = \langle T_c R(h), T_c R(k) \rangle_{L^2}
\end{equation*}
 on \(\mathcal{P}_*\). Moreover \(G\) will be a \emph{first order Sobolev metric}, defined by the arch length integral
\begin{equation*}
  G_c(h,k) = \int_I \langle D_s h^\perp ,D_s^\perp k \rangle+\frac{1}{4}\langle D_s h,v\rangle \langle D_s k,v\rangle \,ds,
\end{equation*}
where \(ds = \vert c' \vert\,dt\), \(v = D_s c = \frac{c'}{\vert c'\vert}\) is the curve of unit length tangents of \(c\), and \(D_s h^\perp = D_s h  - \langle D_s h,v\rangle v\) is the projection of \(D_s h\) onto the space of curves orthogonal to \(v\). 

The space \(C^{\infty}(I, \R^d \setminus \{0\}) \) is a non-convex subset of an inner product space. Therefore, when the SRVT form of two curves \(R(c_1), R(c_2)\) belong to the same convex subset of \(C^{\infty}(I, \R^d \setminus \{0\})\), then the shortest path between them will be the line 
\begin{equation}
  \eta(t) = R(c_2)t +  R(c_1)(1-t). 
\end{equation} 
Therefore, the geodesics from \(c_1\) to \(c_2\) is given by
\begin{equation*}
  \gamma(t) = R^{-1}[R(c_2)t +  R(c_1)(1-t)],
\end{equation*}
and the length of \(\gamma\) will be the equal to \(d_{\mathcal{P}_*(c_1, c_2)}\). For curves which have an SRV form not connected by a line in \(C^{\infty}(I, \R^d \setminus \{0\})\) (e. g. \(c_1 = -c_2\)) the situation is more complicated.  Either the geodesic of the metric \(G\) will not exist, or the geodesic distance will not correspond to the distance \(d_{\mathcal{P}_*}\). 

When \(d \geq 3\),  we can make lines in \(C^{\infty}(I, \R^d )\) smoothly into paths in \(C^{\infty}(I, \R^d \setminus \{0\}\) by arbitrary small perturbations \cite{bauer2014_rprop}[Section 2.1]. Thus, the distance given by \(d_{\mathcal{P}_*}\) will be the geodesic distance of \(G\). However, when \(d \leq 2 \), these smooth perturbations are not always possible, and \(d_{\mathcal{P}_*}\) will not always agree with the geodesic distances induced by \(G\).  As a remedy for \(d \leq 2 \), it is possible to consider geodesic completions of \(\mathcal{P}_* \) as discussed in \cite{bruveris1016_srvtexample}[Section 2.2] and more generally in \cite{bruveris2014_geocomp}.


To extend the Riemannian geometry of \(\mathcal{P}\) to the shape space, we note by \cite{celledoni2016}that the SRVT has the equivariance property:
\begin{equation*}
  R(c \circ \varphi) = \sqrt{\varphi'}R(c) \circ \varphi \quad \forall \ \varphi \in \text{Diff}^+, c \in \mathcal{P}.
\end{equation*}
Therefore, we deduce by integration or \cite[Theorem 3.1]{bauer2014_rprop} that the metrics \(G\) and \(d_{\mathcal{P}_*}\) are reparametrization invariant. The reparametrization invariant distance can then be extended to \(\mathcal{S}_* : = \{[c] \in \mathcal{S}: c(0)=0\}\) by \eqref{eq:distance_inf}. Consequently, finding distances and geodesics between two shapes \([c_1], [c_2]\in \mathcal{S}_*\) has been reduced to finding a reparametrization \(\varphi \in \DiffI\) that minimizes
\begin{equation} \label{eq:srvt_reparam}
  \norm{R(c_1) - \sqrt{\varphi'} R(c_2)\circ \varphi}_{L^2}.
\end{equation}

The optimal solution to the optimization problem above will not necessarily exist in \(\DiffI\). Examples of curves that lead to optimal reparametrization outside \(\DiffI\) can be found in \cite[p.11]{bauer2015why} and \cite[Section 3.2]{woien2019}. One problem is that optimal solutions might have vanishing derivatives. Thus, the optimal solution is not guaranteed to be a diffeomorphism. Still, there are metrics on \(\mathcal{P}\) for which the optimal diffeomorphism is guaranteed as shown in \cite{bauer2014overview}. \citeauthor{bruveris1016_srvtexample} \cite{bruveris1016_srvtexample} instead expands the search space to
\begin{equation*}
  \overline \Gamma = \{\gamma : I \rightarrow I : \gamma \text{\ abs. \ cont. }, \gamma(0) = 0, \gamma(1) = 1, \gamma' \geq 0 \ \text{a.e.}\}.
\end{equation*}
The existence problem is then remedied by showing that for curves \(c_1, c_2 \in C^2(I,\R^d)\) there exists optimal reparametrizations \(\varphi_1^*, \varphi_2^* \in \overline{ \Gamma}\) that minimizes
\begin{equation*}
  d_{\mathcal{P}_*}(c_1 \circ \varphi_1, c_2 \circ \varphi2).
\end{equation*}


\subsection{Shape analysis on Lie groups}\label{subsec:shape-lie}
In computer graphics, the motion of characters can be represented using \emph{Skeletal animation}. This model consists of a rooted tree, where each edge represents a bone, and each node represents a joint. Each node has a local coordinate system related to its parent node via rotation and translation. The global position of each joint is therefore determined by iteratively composing the transformations associated with the bones connecting it to the root joint. Moreover, since each translation will be an element of the \(SE(3)\), all possible model poses will be an element in the \emph{joint space}
\begin{equation}
  \mathcal{J} = SE(3)^d, 
\end{equation}
where \(d\) is the number of bones in the model.

Furthermore, the motions of these skeleton animations will be curves in the joint space over some interval \(I\). Moreover, since all human bones are of fixed length, all the transformations between joints can be represented by a rotation \(SO(3)\). Thus, the joint space of a human skeleton will be the Lie group \(SO(3)^d\). 

To analyse curves in the Lie group \(G\), we first define our new pre-shape space 
\begin{equation*}
  \mathcal{P}:= \text{Imm}(I, G^d) = \{c \in C^{\infty}(I, G^d) :  c'(t) \neq  0 \ \forall \ t \in I  \}. 
\end{equation*}
We then use an extension to the SRVT introduced by \citeauthor{celledoni2016} \cite{celledoni2016}. This transformation is given by 
\begin{equation*}
  \mathcal{R}: \mathcal{P} \rightarrow C^\infty(I, \mathfrak{g}\setminus \{0\} ), \quad \mathcal{R} := \frac{\delta^r(c)}{\sqrt{\norm{\delta^r(c)}}}, 
\end{equation*}
where \(\mathfrak{g}\) is the Lie algebra of \(G\) and \(\delta^r\) is the \emph{right logarithmic derivative} as defined in \cite{right_log_der}[p. 404]. For our purposes, we note by \cite{nice_form}[p.72] that for subgroups of $GL(n)$ the right logarithmic derivative is given by 
\begin{equation*}
   \delta^r(c)(t) = c'(t) \cdot c(t)^{-1}. 
\end{equation*}

Notably, \(\mathcal{R}\)  maps curves from \(\mathcal{\mathcal{P}}\) to curves in the lie algebra \(\mathfrak{g}\). Moreover, \(\mathcal{R}\) is translation invariant \cite{celledoni2016}[Lemam 3.6], meaning that for the right translation given by \(R_{g_1}(g_2) = g_1 \cdot g_2\),  we have 
\begin{equation*}
  \mathcal{R}(c) = \mathcal{R}(R_g \circ c) \forall g, \in G, c in \mathcal{P}.
\end{equation*}
Thus, we define as before the space of curves staring at the identity \(\mathcal{P}_* \{c \in \mathcal{P} : c(0)=e\}\).  

\citeauthor{celledoni2016}  also proved that \(\mathcal{R}\) has many of the same properties as the SRVT defined in \cite{srivastava2011_srvt}. Firstly, by \cite{celledoni2016}[Theorem 3.16], for \(\text{dim} \ \mathfrak{g} > 2\), the distance function 
\begin{equation*}
  d_{\mathcal{P}_*}(c_1, c_2) = \norm{R(c_1) - R(c_2)}_{L^2}, 
\end{equation*}
will be a metric on \(\mathcal{P}_*\) and corresponds to the geodesic distance induced by the elastic metric given in \cite{celledoni2016}[Theorem 3.11]. Moreover since \(\mathcal{R}\) is reparametrization equivariant by \cite{celledoni2016}[Lemma 3.6], then \(d_{\mathcal{P}_*}\) defines a well defined distance function on \(\mathcal{S} / \ / \DiffI\) by \eqref{eq:distance_inf}. 
% Before we can define the SRVT for Lie groups we need the \emph{right logarithmic derivative} given by 
 % \begin{equation*}
%     \delta^r : C^\infty(I, G) \rightarrow  C^\infty(I, G), \quad c \mapsto (R_c)^{-1}_*(c').
% \end{equation*}
% Here \(R_c\) denotes right translation, i. e. \(R_g_1(g_2) = \g_1 \cdot \g_2)\), and Maurer–Cartan 

If a continuous curve \(c\) is given only by data at discrete times \((t_i)_{i=0}^n\). Then a continuos curve \(\overline{c}\) can be created by interpolate along geodesics in SO(3) between each \(c(t_i)\) and \(c(t_{i+1}\), \(0 \leq i \leq n-1\) \cite{geodesic_interpl}.  Then, by \cite{celledoni2016}[p. 23], the SRV form of \(\overline{c}\) will be the piecewise constant curve 
\begin{equation*}
  \overline q(t) = \one_{[t_i, t_{i+1})}\frac{\eta_i}{\sqrt{\norm{\eta_i}}},
\end{equation*}
where
\begin{equation*}
  \eta_i = \frac{\log(c_{i+1}c_{i}^T)}{t_{i+1} -  t_{i}}.
\end{equation*}
