\subsection{Shape analysis on Lie groups}\label{subsec:shape-lie}
In computer graphics, the motion of characters can be represented using \emph{Skeletal animation}. This model consists of a rooted tree, where each edge represents a bone, and each node represents a joint. Each node has a local coordinate system related to its parent node via rotation and translation. The global position of each joint is therefore determined by iteratively composing the transformations associated with the bones connecting it to the root joint. Moreover, since each translation will be an element of the \(SE(3)\), all possible model poses will be an element in the \emph{joint space}
\begin{equation}
  \mathcal{J} = SE(3)^d, 
\end{equation}
where \(d\) is the number of bones in the model.

Furthermore, the motions of these skeleton animations will be curves in the joint space over some interval \(I\). Moreover, since all human bones are of fixed length, all the transformations between joints can be represented by a rotation \(SO(3)\). Thus, the joint space of a human skeleton will be the Lie group \(SO(3)^d\). 

To analyse curves in the Lie group \(G\), we first define our new pre-shape space 
\begin{equation*}
  \mathcal{P}:= \text{Imm}(I, G^d) = \{c \in C^{\infty}(I, G^d) :  c'(t) \neq  0 \ \forall \ t \in I  \}. 
\end{equation*}
We then use an extension to the SRVT introduced by \citeauthor{celledoni2016} \cite{celledoni2016}. This transformation is given by 
\begin{equation*}
  \mathcal{R}: \mathcal{P} \rightarrow C^\infty(I, \mathfrak{g}\setminus \{0\} ), \quad \mathcal{R} := \frac{\delta^r(c)}{\sqrt{\norm{\delta^r(c)}}}, 
\end{equation*}
where \(\mathfrak{g}\) is the Lie algebra of \(G\) and \(\delta^r\) is the \emph{right logarithmic derivative} as defined in \cite{right_log_der}[p. 404]. For our purposes, we note by \cite{nice_form}[p.72] that for subgroups of $GL(n)$ the right logarithmic derivative is given by 
\begin{equation*}
   \delta^r(c)(t) = c'(t) \cdot c(t)^{-1}. 
\end{equation*}

Notably, \(\mathcal{R}\)  maps curves from \(\mathcal{\mathcal{P}}\) to curves in the lie algebra \(\mathfrak{g}\). Moreover, \(\mathcal{R}\) is translation invariant \cite{celledoni2016}[Lemam 3.6], meaning that for the right translation given by \(R_{g_1}(g_2) = g_1 \cdot g_2\),  we have 
\begin{equation*}
  \mathcal{R}(c) = \mathcal{R}(R_g \circ c) \forall g, \in G, c in \mathcal{P}.
\end{equation*}
Thus, we define as before the space of curves staring at the identity \(\mathcal{P}_* \{c \in \mathcal{P} : c(0)=e\}\).  

\citeauthor{celledoni2016}  also proved that \(\mathcal{R}\) has many of the same properties as the SRVT defined in \cite{srivastava2011_srvt}. Firstly, by \cite{celledoni2016}[Theorem 3.16], for \(\text{dim} \ \mathfrak{g} > 2\), the distance function 
\begin{equation*}
  d_{\mathcal{P}_*}(c_1, c_2) = \norm{R(c_1) - R(c_2)}_{L^2}, 
\end{equation*}
will be a metric on \(\mathcal{P}_*\) and corresponds to the geodesic distance induced by the elastic metric given in \cite{celledoni2016}[Theorem 3.11]. Moreover since \(\mathcal{R}\) is reparametrization equivariant by \cite{celledoni2016}[Lemma 3.6], then \(d_{\mathcal{P}_*}\) defines a well defined distance function on \(\mathcal{S} / \ / \DiffI\) by \eqref{eq:distance_inf}. 
% Before we can define the SRVT for Lie groups we need the \emph{right logarithmic derivative} given by 
 % \begin{equation*}
%     \delta^r : C^\infty(I, G) \rightarrow  C^\infty(I, G), \quad c \mapsto (R_c)^{-1}_*(c').
% \end{equation*}
% Here \(R_c\) denotes right translation, i. e. \(R_g_1(g_2) = \g_1 \cdot \g_2)\), and Maurer–Cartan 

If a continuous curve \(c\) is given only by data at discrete times \((t_i)_{i=0}^n\). Then a continuos curve \(\overline{c}\) can be created by interpolate along geodesics in SO(3) between each \(c(t_i)\) and \(c(t_{i+1}\), \(0 \leq i \leq n-1\) \cite{geodesic_interpl}.  Then, by \cite{celledoni2016}[p. 23], the SRV form of \(\overline{c}\) will be the piecewise constant curve 
\begin{equation*}
  \overline q(t) = \one_{[t_i, t_{i+1})}\frac{\eta_i}{\sqrt{\norm{\eta_i}}},
\end{equation*}
where
\begin{equation*}
  \eta_i = \frac{\log(c_{i+1}c_{i}^T)}{t_{i+1} -  t_{i}}.
\end{equation*}
