\documentclass[a4paper]{article}

\usepackage[utf8]{inputenc} %- Løser problem med å skrive andre enn engelske bokstaver f.eks æ,ø,å.
\usepackage[T1]{fontenc} %- Støtter koding av forskjellige fonter.
\usepackage{textcomp} % Støtter bruk av forskjellige fonter som dollartegn, copyright, en kvart, en halv mm, se http://gcp.fcaglp.unlp.edu.ar/_media/integrantes:psantamaria:latex:textcomp.pdf
\usepackage{csquotes}
\usepackage{url} % Gjør internett- og e-mail adresser klikkbare i tex-dokumentet.
\usepackage{hyperref} % Gjør referansene i tex-dokumentet klikkbare, slik at du kommer til referansen i referanselista.
\usepackage[english]{babel} % Ordbok. Hvis man setter norsk i options til usepackage babel kan man bruke norske ord.
\usepackage{amsmath} 				% Ekstra matematikkfunksjoner.
\usepackage{amssymb}
\usepackage{amsfonts}
\usepackage{amsthm}
\usepackage{mathrsfs}
\usepackage{mathtools}
\usepackage{geometry}
\usepackage{tikz-cd}
\usepackage{graphicx}
\usepackage{changepage}
\usepackage{subcaption}
\usepackage{placeins}
\usepackage{bm}
\usepackage{physics}
\usepackage{siunitx}					% Må inkluderes for blant annet å få tilgang til kommandoen \SI (korrekte måltall med enheter)
  \sisetup{exponent-product = \cdot}      	% Prikk som multiplikasjonstegn (i steden for kryss).
   \sisetup{output-decimal-marker  =  {,}} 	% Komma som desimalskilletegn (i steden for punktum).
   \sisetup{separate-uncertainty = true}   	% Pluss-minus-form på usikkerhet (i steden for parentes). 
\usepackage{booktabs} % For å få tilgang til finere linjer (til bruk i tabeller og slikt).
\usepackage[font=small,labelfont=bf]{caption}		% For justering av figurtekst og tabelltekst.
\usepackage[backend=biber]{biblatex}
\addbibresource{./ref.bib}

% math stuff
\newcommand{\restr}[2]{\ensuremath{\left.#1\right|_{#2}}}

% my personal commands
\newcommand{\R}{\mathbb{R}}

%\clearpage % Bruk denne kommandoen dersom du vil ha ny side etter det er satt plass til figuren.
% Disse kommandoene kan gjøre det enklere for LaTeX å plassere figurer og tabeller der du ønsker.
\setcounter{totalnumber}{5}
\renewcommand{\textfraction}{0.05}
\renewcommand{\topfraction}{0.95}
\renewcommand{\bottomfraction}{0.95}
\renewcommand{\floatpagefraction}{0.35}

% math stuff

\newtheorem{definition}{Definition}[section]
\newtheorem{theorem}{Theorem}[section]
\newtheorem{claim}[theorem]{Claim}
\newtheorem{proposition}[theorem]{Proposition}
\newtheorem{lemma}[theorem]{Lemma}
\newtheorem{corollary}[theorem]{Corollary}
\newtheorem{conjecture}[theorem]{Conjecture}
\newtheorem*{observation}{Observation}
\newtheorem*{example}{Example}
\newtheorem*{remark}{Remark}

\graphicspath{{../}}

\title{Specialization Project}

\author{Alexander Johan Arntzen }

\date{\today}

%%%%%%%%%%%%%%%%%%%%%%%%%%%%%%%%%%%%%%%%%%%%%%%%%%%%%%%%%%%%%%%%%%%%%%%%%
\begin{document}



\maketitle

\section{Theory}
\subsection{Lie groups}
In this subsection some basic Lie grup properties are introduced. Based on \cite{celledoni2016}. Basic knowlegde of smooth manifolds is assumed. 
\begin{definition}[Lie group]
  A Lie group $G$ is a smooth manifold that is also a group, such that multiplication
  \begin{equation}
    \mu : G \times G \rightarrow G  
  \end{equation}
  and inversion
  \begin{equation}
    i : G  \rightarrow G  
  \end{equation}
  are smooth. 
\end{definition}
Since both  multiplication and inversion are smooth we define multiplication by $g \in G$ as diffeomorphism.
\begin{definition}
  Let $G$ be a Lie group and $g \in G$. Then the \textbf{right translation} by $g$ is defined as
  \begin{equation}
    r_g(h):= gh \quad  \forall \ h  \in G.
  \end{equation}
\end{definition}
Once translation is defined, the following property of vector fields follows. 
\begin{definition}
  A vector field $X$ on a Lie group $G$ is \textbf{right invariant} if it is invariant under the pushforward of all right translation. That is is
  \begin{equation}
    (r_g)_*X = X  \quad \forall g \in G 
  \end{equation}
\end{definition}
An invariant vector field $X$ on a Lie group $G$ is thus uniquely defined by the vector field's value at the identity element $e$. Similarly, for all vectors $\xi \in T_eG$ we can define an invariant vector field  $X^{\xi}$ on $G$. 
\begin{definition}[Lie algebra]
  Let $e$ be the identity element of the Lie group $G$. Then the \textbf{Lie algebra} $\mathfrak{g}$ is the vector space $T_eG$ together with a lie bracket as bilinear product.  
\end{definition}
% TODO: write about bracket 

\subsection{The exponential mpa}

\begin{definition}[Exponential map]
Let $\mathfrak{g}$ be the Lie algebra of the Lie group $G$ with identity $e$. For all $\xi \in \mathfrak{g}$ let $\gamma^{\xi}$ be the integral curve of the right invariant vector field $X^{\xi}$ where $X^\xi (e)=\xi$ and $\gamma^\xi(0) = e$. The \textbf{exponential map} is the map $\exp : \mathfrak{g} \rightarrow G$ such that $\exp(\xi) = \gamma^\xi(1)$. 
\end{definition}
% TODO: make an argumet for why it is well defined
\begin{proposition}
  Let $X$ be a right invariant vector field on the Lie group $G$ with identity $e$. Then the flow of $X$, $\phi_t$ is given by $\phi_t(g) = \exp(tX(e))g$.
\end{proposition}
\begin{proof}
  \item one parameter subgoup. 
  \item diffeomorphism pullback invarinat, flow. 
\end{proof}
% TODO: show that the flow is defined for all times t 

\cite{lystad2019}
% \begin{figure}[t]
%     \centering
%     \includegraphics[width=0.8\textwidth]{figures/task5/final_optim_points.pdf}
%     \caption{Optimal curve of $(D^*,v^*)$ values. The initial Sobol points are also shown with the corresponding $G$ value}
%     \label{fig:task5}
% \end{figure}

\section*{Appendix}
The code for this project can be found at \url{https://github.com/alexarntzen/prosjektoppgave}


\printbibliography
\end{document}

