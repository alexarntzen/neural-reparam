
\usepackage[utf8]{inputenc} %- Løser problem med å skrive andre enn engelske bokstaver f.eks æ,ø,å.
\usepackage[T1]{fontenc} %- Støtter koding av forskjellige fonter.
\usepackage{textcomp} % Støtter bruk av forskjellige fonter som dollartegn, copyright, en kvart, en halv mm, se http://gcp.fcaglp.unlp.edu.ar/_media/integrantes:psantamaria:latex:textcomp.pdf
\usepackage{csquotes}
\usepackage{url} % Gjør internett- og e-mail adresser klikkbare i tex-dokumentet.
\usepackage[colorlinks]{hyperref} % Gjør referansene i tex-dokumentet klikkbare, slik at du kommer til referansen i referanselista.
\usepackage[norsk,english]{babel} % Ordbok. Hvis man setter norsk i options til usepackage babel kan man bruke norske ord.
\usepackage{amsmath} 				% Ekstra matematikkfunksjoner.
\usepackage{amssymb}
\usepackage{amsfonts}
\usepackage{amsthm}
\usepackage{mathrsfs}
\usepackage{mathtools}  
\usepackage{geometry}
\usepackage{tikz-cd}
\usepackage{graphicx}
\usepackage{svg}
\usepackage{changepage}
\usepackage{subcaption}
\usepackage{placeins}
\usepackage{bm}
\usepackage{physics}
\usepackage{siunitx}					% Må inkluderes for blant annet å få tilgang til kommandoen \SI (korrekte måltall med enheter)
  \sisetup{exponent-product = \cdot}      	% Prikk som multiplikasjonstegn (i steden for kryss).
   \sisetup{output-decimal-marker  =  {,}} 	% Komma som desimalskilletegn (i steden for punktum).
   \sisetup{separate-uncertainty = true}   	% Pluss-minus-form på usikkerhet (i steden for parentes). 
\usepackage{booktabs} % For å få tilgang til finere linjer (til bruk i tabeller og slikt).
\usepackage[font=small,labelfont=bf]{caption}		% For justering av figurtekst og tabelltekst.
\usepackage[
backend=biber,
]{biblatex}
\addbibresource{ref.bib}

% math stuff
\newcommand{\restr}[2]{\ensuremath{\left.#1\right|_{#2}}}

% my personal commands
\newcommand{\R}{\mathbb{R}}

%\clearpage % Bruk denne kommandoen dersom du vil ha ny side etter det er satt plass til figuren.
% Disse kommandoene kan gjøre det enklere for LaTeX å plassere figurer og tabeller der du ønsker.
\setcounter{totalnumber}{5}
\renewcommand{\textfraction}{0.05}
\renewcommand{\topfraction}{0.95}
\renewcommand{\bottomfraction}{0.95}
\renewcommand{\floatpagefraction}{0.35}

% math stuff

\newtheorem{definition}{Definition}[section]
\newtheorem{theorem}{Theorem}[section]
\newtheorem{claim}[theorem]{Claim}
\newtheorem{proposition}[theorem]{Proposition}
\newtheorem{lemma}[theorem]{Lemma}
\newtheorem{corollary}[theorem]{Corollary}
\newtheorem{conjecture}[theorem]{Conjecture}
\newtheorem*{observation}{Observation}
\newtheorem*{example}{Example}
\newtheorem*{remark}{Remark}
